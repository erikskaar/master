\section{Motivation}\label{sec:motivation}

In the recent years, IoT has become more and more prevalent in our society. This has lead to an increase in the data generated every day, especially when it comes to spatial data. Companies like Strava have earned millions of dollars generating, storing and aggregating spatial data for their users \cite{strava_worth}. This data is often presented as trajectories in the form of linked directional points in two or three dimensions showing the path of the user. Spatial databases have long been able to effectively query data on their spatial properties, but there does not seem to be much research on utilizing the linked directional properties of the trajectories for querying the data. This defines the motivation of this project: make queries of linked directional spatial data more effective.
\section{Goal}
The goal for this thesis is to test if the data structure presented is a viable data structure for storing spatial data with linked directional properties.


\section{Objectives}


To reach the goal, one would have to fulfill a set of objectives, which can be seen as subgoals for this thesis.
\begin{enumerate}
	\item Implement the data structure in a programming language
	\item Setup a benchmarking environment complete with baseline performances in other data structures to compare it with.
\end{enumerate}

By completing these objectives the results can be compared and discussed, answering if the data structure compares against existing alternatives in querying spatial data with directional propertes.

\section{Chapter overview}

This paper will be divided into chapters, with the first chapter containing the background where existing data structures and indexing methods will be presented, discussed and compared. The context and gap in knowledge will be explained more in-depth. Related work will also be presented here. Chapter three encompasses the methodology used in this project. Here the new data structure will be presented, as well as how the benchmarking has been conducted, including the choice of technology and comparison data structures. In chapter four we will present the results of the benchmarking and look at how they stack up against the predicted outcome. Chapter five contains future work and chapter six contains the conclusion.
\\

\textbf{Disclaimer:} chapter one may contain text already used in \textit{IT3915} as that is the preliminary project for this thesis and is closely related.
